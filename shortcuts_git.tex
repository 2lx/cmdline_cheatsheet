\documentclass[main.tex]{subfiles}

\lstset{
    morekeywords={add, checkout, clone, commit, config, diff, fetch, help, init, log, mv, pull, push, reset, remote, rm, status, tag}
}

\newcommand{\git}{\lstinline{git}\xspace}

\begin{document}

\git configuration files priorities:
\begin{enumerate}
\item local repository settings: \path{.git/config}
\item user settings: \path{$HOME/.gitconfig} or \path{$HOME/.config/git/config}
\item system settings: \path{/etc/gitconfig}
\end{enumerate}

There are 3 \git directories:
\begin{itemize}
\item working (current state)
\item index (\path{.git/index})
\item repository
\end{itemize}

There are 4 file states in \git working directory:
\begin{itemize}
\item untracked files
\item tracked (non-modified) files
\item modified files
\item indexed files
\end{itemize}

\setlength\LTleft{0pt}
\setlength\LTright{0pt}
\begin{longtable}{l | l}
\multicolumn{2}{l}{ \lstinline|git config| } \lstinline|$> git help config|  \\
    \hline
\lstinline|--global user.name ""| & \multirow{3}{*}{ Update a global config parameter} \\
\lstinline|--global user.email ""| & \\
\lstinline|--global core.editor ""| & \\
\hline
\end{longtable}

All possible commands:
\begin{lstlisting}
$> git add/checkout/clone/commit/config/diff/fetch/help/init/log/mv/pull/push/reset/remote/rm/status/tag
\end{lstlisting}

Examples:
\begin{lstlisting}
$> git status -s # short format of commit logs

$> git diff # compares WORK with INDEX
$> git diff --staged / diff --cached # compare INDEX with repo HEAD

$> git commit -a # adds to index all modified files
$> git commit --amend # recommits last commit

$> git log -p -2 # shows all diffs from last 2 commits
$> git log --stat # shows short statisctics for every commit
$> git log --pretty=format:"..." --since --until --grep --author

$> git reset HEAD filename # unstages file (delete from INDEX)

$> git checkout -- [filename] # discards changes in working directory
# -- splits file names section

$> git remote -v # shows all remote repos
$> git remote add name url
$> git remote rename nameold namenew
$> git remote rm
$> git remote show origin # shows repo state
$> git remote set-url name url # sets new url for this name

$> git tag # adds lightweight tag
$> git tag -a # adds tag with comment and shapshot
\end{lstlisting}

\end{document}
